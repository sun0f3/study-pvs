%% Преамбула TeX-файла

% По госту положен 14-ый шрифт, однако мы возьмём 12-ый.

% 1. Стиль и язык
%\documentclass [12pt,oneside]{rusthesis} % Стиль (по умолчанию 14pt)
\documentclass[pdftex,12pt,a4paper]{report}

\usepackage[T2A]{fontenc}
\usepackage[utf8]{inputenc}
\usepackage[english, russian]{babel}

% 2. Обязательные вещи
\renewcommand{\labelitemi}{\normalfont\bfseries{--}} % список без номеров
\newcommand{\HRule}{\rule{\linewidth}{0.5mm}}
\newcommand{\code}[1]{\texttt{#1}}

\sloppy

% 3. Добавляем гипертекстовое оглавление в PDF
\usepackage[
bookmarks=true, colorlinks=true, unicode=true,
urlcolor=black,linkcolor=black, anchorcolor=black, c
itecolor=black, menucolor=black, filecolor=bla,
]{hyperref}


% Прочее, дополнять по вкусу
\usepackage{graphicx}	% Пакет для включения рисунков

%\usepackage{makecell}
%\usepackage{multirow}
%\usepackage{ulem}
%\usepackage{indentfirst}

%\usepackage[pdftex]{graphicx}
%\graphicspath{{pic/}}
%\usepackage{tikz}

% Увы, поля придётся уменьшить из-за листингов.
\topmargin -1cm
\oddsidemargin -0.5cm
\evensidemargin -0.5cm
\textwidth 17cm
\textheight 24cm

% Картнки и tikz
%\usetikzlibrary{snakes,arrows,shapes}



\Russian 
